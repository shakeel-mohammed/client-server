\documentclass[a4paper]{article}
\usepackage[utf8]{inputenc}

\begin{document}

\title{COMP3100 project report}
\hrule \medskip % Upper rule
\begin{minipage}{0.9\textwidth}
\centering 
\large % Title text size
Stage 1 COMP3100 Distributed Systems, S2, 2022\\
\normalsize % Subtitle text size
SID: 45334625, Name: Shakeel Mohammed
\end{minipage}
\medskip\hrule % Lower rule
\bigskip

\section{Introduction}
What is this project?
This report presented is to describe the overview, design, and implementation of a scheduling system that implements the Largest Round Robin algorithm whilst distributing jobs to servers in a simulated distributed system (provided by MQ University). The results generated by the use of this project are to be compared to a reference implementation (also provided by MQ University).

At it's core, a distributed system is a system that has various components which are spread across a network. This provides many advantages such as efficiency and redundancy, it also introduces complications such as an increased complexity, synchronisation and replication issues.

In a distributed system, the load is distributed across multiple machines to achieve fast and more reliable results. Each distributed system requires a component to orchestrate and distribute each request.

This project acts as an orchestrating component of a distributed system.

The goal for this project is to have a client which connects to the server, and makes scheduling decisions based on the desired algorithm.

\section{System Overview}
\label{sec:section2}
The system is comprised of two main components:
\begin{enumerate}
  \item the ds-sim Server - initiates the request to process each job, as well as simulates a distributed system.
  \item the Client (this project) - handles the request to process each job and schedules jobs for processing.
\end{enumerate}

[little diagram of system overview here]

\subsection{Communication Protocol}
The two components communicate via a Socket, which uses TCP at the network layer. See here \TODO cite[https://docs.oracle.com/javase/tutorial/networking/sockets/definition.html]

The protocol is as follows:
\begin{enumerate}
  \item the ds-sim server is started, creates a simulated system based on the confile file passed in, and waits for incoming socket connections.
  \item the client is started and initiates a handshake with the server.
  \item the server writes information on the simulated system to a ds-system.xml file.
  \item in a loop, the following happens:
  \begin{enumerate}
      \item the client sends REDY command.
      \item the server responds with either:
      \begin{enumerate}
          \item the server responds with JOBN (new job), and the client finds and selects a server capable to processing the job based on the Largest Round Robin algorithm.
          \item the server responds with JCPL (completed), and the client ignores the job.
          \item the server responds with an error, and the client logs the error and stops the job scheduling process.
          \item the server responds with NONE, this breaks the loop.
      \end{enumerate}
  \end{enumerate}
  \item once the loop is complete (no more jobs left to process), the client sends QUIT command.
  \item the server replies with QUIT, and the connection is closed gracefully.
\end{enumerate}


\section{Design}
\label{sec:section3}
The design must cater for the connection between the two main components; the server and the client, as well as break down the current state of the simulated ds-sim system at any one time to handle incoming jobs. The two main entities in ds-sim are Servers and Jobs. These are the entities that are used to make scheduling decisions.

\subsection{Servers}
A server is a compute resource equipped with its own CPU, memory, and disk. They can be either physical or virtual servers. In the case of the ds-sim system, each simulated server is virtual. Each server has it's own serverId, serverType, limit, bootUpTime, hourlyRate, cores, memory, and disk attributes.

\subsection{Jobs}
A Job can represent a task, or a single linux process. Each job has it's own jobId, type, submitTime, estimatedRunTime, cores, memory, and disk attributes.

\subsection{Scheduling}
Processing information on each of these entities is crucial for our design to be able to achieve our goal to schedule jobs based on Largest Round Robin, which operates by scheduling jobs to servers that are of the type in the system have the highest number of cores, in a round robin manner. i.e. first job to server 1, second job to server 2, and so on.

FIND A SMALL SNIPPET OF LRR and place here
\subsection{Components}
The design achieves job scheduling by Largest Round Robin by the use of the following components:

\subsubsection*{ConfigDataLoader}
A singleton object which serves as a configuration parameter store. It reads key/value pairs from the config.properties file. It allows the client to be re-configured without requiring re-compilation.

\subsubsection*{ClientServerConnection}
Handles the client/server socket connection. It provides an interface for the rest of the application to send and receive messages to the ds-sim server.

\subsubsection*{Orchestrator}
Responsible for making scheduling decisions. It currently implements the Largest Round Robin algorithm. It can easily be extended to support additional algorithms.

\subsubsection*{SimulatedSystem}
Responsible for containing information on the simulated system which the ds-sim server provides. It provides information on the system as a swarm of servers.

\subsubsection*{SimulatedServer}
Represents one server which exists in the ds-sim system. It contains information on the type of server, number of cores, etc.

\subsubsection*{Job}
Represents one job which exists in the ds-m system. It contains all of the attributes that define a job. A job is instantiated by the use of a JobInformation and a JobInformation component as they allow for a standardization layer. This is required because the ds-sim server is capable to providing a job in multiple different formats.

\subsection{Constraints}
This project requires a Java 1.8 to be run, which can be downloaded by following the instructions here, [https://docs.datastax.com/en/jdk-install/doc/jdk-install/installOpenJdkDeb.html]

\subsection{Considerations}
As per the ds-sim protocol, there are two methods by which a connecting client could retrieve information about the simulated system:
\begin{enumerate}
  \item ds-system.xml file - once a client has connected and a handshake is successful, the ds-sim server will create a ds-system.xml file which contains information on the simulated system in an xml format. This can then be parsed by the client used to make scheduling decisions
  \item GETS command - a client is able to query the ds-sim server for a list of servers (GETS AVAIL and GETS CAPABLE). This can be a complete exhaustive list containing all of the simulated servers that can eventually become available, or a more precise list of server capable of handing a job based on cpu, memory and disk, which are immediately available for use.
\end{enumerate}

This design uses the GETS CAPABLE command to increase efficiency and search only for the servers that are capable of handling the particular job ready for processing. 


\section{Implementation}
\label{sec:section4}
\subsection{Technologies}
Java 1.8 - Java is a programming language and computing platform first released by Sun Microsystems in 1995. It has evolved from humble beginnings to power a large share of today’s digital world. This project has been created using Java 1.8.

\subsection{Libraries}
The java client employs the use of the following classes provided by Java 1.8:
\begin{enumerate}
  \item Socket - A socket is an endpoint for communication between two machines. A socket is used for the client/server connection. The client/server connect via a Socket. This Socket connection is established and persisted throughout the entire run.
  \item DataInputStream - A data input stream lets an application read primitive Java data types from an underlying input stream in a machine-independent way. The client/server connection employs a DataInputStream to read responses from the server.
  \item File - An abstract representation of file and directory pathnames. The ConfigDataLoader employs the use of the java File class to create a File object, which represents the config.properties file. This object is then used to read each property.
  \item FileInputStream - A FileInputStream obtains input bytes from a file in a file system. The ConfigDataLoader employs the use of a FileInputStream to read each line from the config.properties file and store them.
  \item Properties - The Properties class represents a persistent set of properties. The Properties can be saved to a stream or loaded from a stream. The ConfigDataLoader employs the user of a Properties object to presist the key/value pairs provided in the config.properties file.
  \item ArrayList - A resizable-array implementation of the List interface. Java Arraylists are utilised by the SimulatedSystem class to keep track of the SimulatedServers within the system, as well as the SimulatedServer class to keep track of the Jobs relating to the SimulatedServer.
\end{enumerate}

\subsection{Job Scheduling}
The implementation as per the ds-sim protocol and LRR is as follows:
[diagram of interaction here]

\section{Usage}
\label{sec:section5}
\subsection{Compile}
A pre-compiled version of this project can be found in the /compiled directory. Otherwise, the compile.bash script can be used to compile if required. From the main directory, run "bash scripts/compile.bash". This will compile the files in the /src directory and overwrite the files in the /compiled directory.

\subsection{Run}
Navigate to the config.properties file and observe the key/values pairs. The host IP, port, algorithm, are configurable from this file. There is also a value for buffer\_for\_record\_length which is the amount of bytes added to the calculated length of buffers used throughout the client. 

Navigate to the /compiled directory and run 'java Main'.

\section{References}

\end{document}